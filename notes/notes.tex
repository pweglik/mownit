\documentclass[12pt,a4paper,openright]{mwrep}

\usepackage{lmodern}
\usepackage[T1]{polski}
\usepackage[utf8]{inputenc}

\usepackage[a4paper,
            tmargin=2cm,
            bmargin=2cm,
            lmargin=2cm,
            rmargin=2cm,
            bindingoffset=0cm]{geometry}

\usepackage{tocloft}
\usepackage{hyperref}

\usepackage{amsmath}
\usepackage{amssymb}
\usepackage{siunitx}

\usepackage{listings}

\usepackage{graphicx}
\usepackage{subfig}
\usepackage{float}
\usepackage{booktabs}

\hypersetup{
    colorlinks,
    citecolor=black,
    filecolor=black,
    linkcolor=black,
    urlcolor=black
}

\newtheorem{definition}{Def}

\begin{document}

\title{%
Metody obliczeniowe w nauce i technice\\
Notatki
}

\author{\\Przemysław Węglik}

\date{28.02.2022}

\maketitle

\tableofcontents



\chapter{Wykład 1 - wprowadzenie, arytmetyka komputerowa}

\section{Wprowadzenie}
Metod numerycznych używamy wtedy kiedy ciężko odnaleźć jest
rozwiązanie analityczne (tz. nie istnieje lub otrzymanie go ma dużą złożoność)
Wyniki są oczywiście przybliżone. Dokładność obliczeń może być z góry określona~i
dobiera się ją w zależności od potrzeb.

\section{Reprezentacja stałopozycyjna}

\section{Reprezentacja stałopozycyjna}

\end{document}
